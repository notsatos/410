\documentclass[9pt,reqno]{amsart}
\usepackage{graphicx}
\usepackage{fullpage}


\usepackage{mathpazo}
\usepackage{euler}


\graphicspath{ {./urpimages/} }
\usepackage[margin=2.5cm]{geometry}
\usepackage{amsfonts,amssymb,latexsym,amsmath, amsthm}
\usepackage{tikz-cd}
\usepackage{mathrsfs}
\usepackage{comment}
\excludecomment{confidential}
\usepackage{enumitem}
\usepackage{caption}
\theoremstyle{definition}
\usepackage[linktocpage=true]{hyperref}
%% this allows for theorems which are not automatically numbered
\newtheorem{defi}{Definition}[section]
\newtheorem{theorem}{Theorem}[section]
\newtheorem{lemma}{Lemma}[section]
\newtheorem{obs}{Observation}
\newtheorem{exercise}{Exercise}[section]
\newcommand{\heg}{\text{Heg}}
\newtheorem{rem}{Remark}[section]
\newtheorem{construction}{Construction}[section]
\newtheorem{prop}{Proposition}[section]
\newtheorem{coro}{Corollary}[section]
\DeclareMathOperator{\spec}{Spec}
\DeclareMathOperator{\im}{im}
\DeclareMathOperator{\obj}{obj}
\DeclareMathOperator{\cl}{cl}
\DeclareMathOperator{\ext}{Ext}
\DeclareMathOperator{\tor}{Tor}
\DeclareMathOperator{\ann}{ann}
\DeclareMathOperator{\id}{id}
\DeclareMathOperator{\gal}{Gal}
\DeclareMathOperator{\coker}{coker}
\newcommand{\degg}{\textup{deg}}
\newtheorem{ex}{Example}[section]
\usepackage{hyperref}
\usepackage{xcolor}
\definecolor{winered}{rgb}{0.5,0,0}
%% The above lines are for formatting.  In general, you will not want to change these.
%%Commands to make life easier
\newcommand{\RR}{\mathbf R}
\newcommand{\aff}{\mathbb A}
\newcommand{\ff}{\mathbb F}
\newcommand{\cccC}{\mathbf C}
\newcommand{\oo}{\mathcal{O}}
% \newcommand{\ZZ}{\mathbf Z}
\newcommand{\pring}{k[x_1, \ldots , x_n]}
\newcommand{\polyring}{[x_1, \ldots , x_n]}
\newcommand{\poly}{\sum_{\alpha} a_{\alpha} x^{\alpha}} 
\newcommand{\ZZn}[1]{\ZZ/{#1}\ZZ}
% \newcommand{\QQ}{\mathbf Q}
\newcommand{\rr}{\mathbb R}
\newcommand{\cc}{\mathbf C}
\newcommand{\complex}{\mathbf {C}_\bullet}
\newcommand{\nn}{\mathbb N}
\newcommand{\zz}{\mathbf Z}
\newcommand{\cat}{\mathbf{C}}
\newcommand{\ca}{\mathbf}
\newcommand{\zzn}[1]{\zz/{#1}\zz}
\newcommand{\qq}{\mathbf Q}
\newcommand{\calM}{\mathcal M}
\newcommand{\latex}{\LaTeX}
\newcommand{\V}{\mathbf V}
\newcommand{\tex}{\TeX}
\newcommand{\sm}{\setminus} 
\newcommand{\dom}{\text{Dom}}
\newcommand{\lcm}{\text{lcm}}
\DeclareMathOperator{\GL}{GL}
\DeclareMathOperator{\Hom}{Hom}
\DeclareMathOperator{\aut}{Aut}
\DeclareMathOperator{\inn}{Inn}
\newcommand{\sym}{\text{Sym}}
\newcommand{\ord}{\text{ord}}
\newcommand{\ran}{\text{Ran}}
\newcommand{\pp}{\prime}
\newcommand{\lra}{\longrightarrow} 
\newcommand{\lmt}{\longmapsto} 
\newcommand{\xlra}{\xlongrightarrow} 
\newcommand{\gap}{\; \; \;}
\newcommand{\Mod}[1]{\ (\mathrm{mod}\ #1)}
\newcommand{\idealp}{\mathfrak{p}}
\newcommand{\rmod}{\textit{R}-\textbf{Mod}}
\newcommand{\idealP}{\mathfrak{P}}
\newcommand{\ideala}{\mathfrak{a}}
\newcommand{\idealb}{\mathfrak{b}}
\newcommand{\idealA}{\mathfrak{A}}
\newcommand{\idealB}{\mathfrak{B}}
\newcommand{\idealF}{\mathfrak{F}}
\newcommand{\idealm}{\mathfrak{m}}
\newcommand{\s}{\mathcal{S}}
\newcommand{\ccc}{\mathfrak{C}}
\newcommand{\idealM}{\mathfrak{M}}
%Itemize gap:

% \pagecolor{black}
% \color{white}
% Author info

\title{Math 410: Notes }
\author{Juan Serratos}
\begin{document}
\maketitle
\section{Groups}
\subsection{Group Actions and Orbit-Stablizer Theorem}
\begin{defi}
Let $G$ be a group and $A$ be a set. Then a (left) \textbf{group action} is a map $\alpha \colon G \times A \to A$ such that the following two axioms are satisfied for all $g, h \in G$ and $a \in A$:
\begin{itemize}
	\item $\alpha (e, x) = x $, and
	\item $\alpha (g, \alpha (h, x)) = \alpha (gh, x)$.
\end{itemize}	
\end{defi}
The usual convention for the notation of a group action is that we will denote it simply using a dot when the context is clear, that is, if we let $G$ act on a set $A$, then $e \cdot x  = x$ and $ g \cdot (h \cdot x)) = (gh) \cdot x$ is the way we denote the two needed axioms above. 
\begin{ex}
The standard example is that if we have $G = S_n$ (or a $G$ be a subgroup of $S_n$ such as $A_n$) and $A = \{ 1, 2, \ldots, n \}$. Then $G$ acts on $A$ by the formula $g \cdot x = g(x)$; that is, we're taking a bijective map $g \colon G \to G$ and $\alpha (g, x) = g(x) \in X$. The first axiom is $e \cdot x = e(x) = x$ is clear as $e \colon G \to G$ is the trivial mapping, and the second axiom is simple: given two bijective maps $g, h \colon G \to G$, then $g \cdot (h \cdot x)) = g \cdot (h(x)) = g(h(x)) = (g \circ h)(x) = (gh)\cdot x$. Furthermore, to make things clear, let $A = \{1, 2,3 \}$. Then if we have the cycle $(1 \; 2 \; 3) $, then $(1 \; 2 \; 3)\cdot x = (1 \; 2 \; 3)(x)$ and so checking all values in $A = \{1, 2, 3 \}$, then $(1 \; 2 \; 3) (1) = 2$, $(1 \; 2 \; 3) (2) = 3$, $(1 \; 2 \; 3) (3) = 1$. For great measure consider the composition of two cycles $( 1 \; 2 \; 3) ( 1 \; 3 \; 2)$ in $S_3$. Then we look towards checking the second axiom: $(1 \; 2 \; 3) \cdot  (1 \; 3 \; 2) \cdot x  = (1 \; 2 \; 3) \cdot (1 \; 3 \; 2) (x)$, and checking all values of $x \in A$, then $(1 \; 3 \; 2) (1) = 3$, $(1 \; 3 \; 2) (2) = 1$, and $(1 \; 3 \; 2) (3) = 2$, and so $ (1 \; 2 \; 3) ((1 \; 3 \; 2) (1)) = (1 \; 2 \; 3) (3) = 1$, $ (1 \; 2 \; 3) ((1 \; 3 \; 2) (2)) = (1 \; 2 \; 3) (1) = 2$, and $ (1 \; 2 \; 3) ((1 \; 3 \; 2) (3)) = (1 \; 2 \; 3) (2) = 3$. Now that we've checked this, we should have that this condides with the other side of this axiom: composing $( 1 \; 2 \; 3) ( 1 \; 3 \; 2)$ gives us $(1) (2) (3) = e$ the identity permutation. Thus $(1) (2) (3) \cdot x = (1) (2) (3)(x)$ gives us $(1) (2) (3) (1) = 1$, $(1) (2) (3) (2) = 2$, and $(1) (2) (3) (3) = 3$. Thus the permutations agree on all values of $x \in A$. The second axiom is sastified for the these two cycles, but we've already done the general case in the first few sentences. 
\end{ex}
\begin{defi}Let $G$ be a group acting on a set $A$. 
\begin{itemize}
	\item A \textbf{fixed point} of an element $g \in G$ is an element $x \in A$ such that $g \cdot x = x$.
	\item The \textbf{stablizer} $G_x$ of a point $x \in X$ is the set of elements $g \in G$ such that $x$ is a fixed point of $g$. 
	\item The \textbf{orbit} of an element $x \in X$ is the set of elements $y \in X$ such that $g \cdot x = y$ for some $g \in G$. 
\end{itemize}
\end{defi}
\begin{rem}
	The of orbit $x$ is denoted by $G \cdot x = \{ g \cdot x \colon g \in G \}$. We say that a group action is \textit{transitive} if and only it has exactly one orbit, that is, if there exists $x \in X$ with $G \cdot x = X$. This is true if and only if $G \cdot x = X$ for all $x \in X$. Or, rather, a group action is transitive if for any two $x, y \in X$ there exits a $g \in G$ such that $g \cdot x =y$. The set of all orbits of $X$ under the group action of $G$ is written as $X/G$, and is called the \textit{quotient} of the action. A $G$-\textit{invariant} element is an element $x \in X$ such that $g \cdot x = x$ for all $g \in G$. The set of all such $x$ is denoted $X_G$ and is called the $G$-invariants of $X$. When $X$ is a $G$-module, i.e. $X$ is an additive abelian group and $G$ is acting on it with the additional action axiom that $g \cdot (a+b) = g \cdot a + g \cdot b$ with $a, b \in X$, then $X^G$ is the zeroth cohomology group of $G$ with coefficients in $X$, that is, $H^0 (G; X) = X^G$.
	
	For each $x \in X$, the \textbf{stabilzer} of $x$ by $G$ is defined as the set $G_x = \{ g \in G \colon g \cdot x = x \}$, and another common notation for this set is $\text{stab} (x)$. Sometimes this set defined as being the ``stabilizer subgroup" of $G$. The reason should be clear: Clearly $G_x \subseteq G$ and $G_x \neq \varnothing$ as $e \in G $ and $e \cdot x = x$ by the action axioms, and so $e \in G_x$. Now take $a, b \in G_x$, then we need to show $ab^{-1} \in G_x$ (recall this is the subgroup test). So, as $a \cdot x = x $ and $b \cdot x = x$, and as $e \cdot x = x$, then $e \cdot x = bb^{-1} \cdot x = b^{-1} b \cdot x = b^{-1} \cdot x = x$; thus $b^{-1} \in G_x$. Then $ab^{-1} \cdot x = a \cdot (b^{-1} \cdot x) = a \cdot x = x$. Therefore $G_x \leq G$. This is typically not a normal subgroup, however. 
	
	The action of $G$ on $X$ is \textit{free} if and only if all stablizers are trivial; that is, for all $k \in X$, we have that $G_k = \{ e_G \}$. An actions is called \textit{faithful} if we have that the intersection of all stablizers is trivial $\{ e_G \}$. Thus, if we let $E = \{G_x \}_{x \in X}$ denote the set of stabilizers for all $x \in X$, then the action on of $G$ on $X$ is free if and only if $\cup E = \{e_G \}$; the action of $G$ on $X$ is called faithful if $\cap E = \{e_G \}$. Furthermore, we can reformulate this as saying an action is faithful if for two distinct elements $g \neq h$ in $G$ there exists an $x \in X$ such that $g \cdot x \neq h \cdot x$; contrapostivitly, if for all $x \in X$, we have $g \cdot x = h \cdot x$, then $g = h$. Moreover, an action is called free if for any two distinct elemenets $g \neq h$ in $G$ and all $x \in X$ we have $g \cdot x \neq h \cdot x$; contrapositivtly, if we have $g \cdot x = h \cdot x$ for some $x \in X$, then $g = h$ in $G$. We can see how related these two notions are to one another. In fact, as you should've seen, a free action is faithful. 
\end{rem}
\begin{prop}
	A group $G$ acts faithfully on $X$ if and only if the homomorphism $G \to S_X$ has a trivial kernel. 
\end{prop}
\begin{proof}
Let $G$ be a group acting on $X$ and let the action be faithfull. For all $g \in G$, define $\sigma_g \colon X \to X$ where $\sigma_g \colon x \mapsto g \cdot x$; this map is a bijection from $X \to X$. Now define the map $\varphi \colon G \to S_X$ by $\varphi \colon g \mapsto \sigma_g$. Take  $g, h \in G$. Then $\varphi (gh) = \sigma_{gh}$, and so take some $x \in X$. So $\sigma_{gh} (x) = gh \cdot x = g\cdot (h \cdot x) = \sigma _g (\sigma_h (x)) = (\sigma_g \sigma_h)(x)$; thus $\varphi(gh) = \sigma _g \sigma_h = \varphi(g) \varphi(h)$ and $\varphi$ is a homomorphism. Now consider the kernel of this maps, i.e. \\ $\ker (\varphi) = \{z \in G \colon \varphi (z) = \sigma_z =\id_X \}$. If we take the trivial $e \in G$ then $\varphi (e) = \sigma_e$, and so on every element $x \in X$, $\sigma_{e} (x) = e \cdot x = x$ by the first action axiom, and we see that $\sigma_e = \id_X$; therefore $e_G \in \ker (\varphi)$. Suppose that we have some other element $q \in \ker (\varphi)$ distinct from $e_G$. Then we have that $\sigma_q = \sigma_e$, i.e. for all $x \in X$, $\sigma_q (x) = q \cdot x = e \cdot x = \sigma_e (x)$, but as the action is faithfull then $q \cdot x = e \cdot x$ implies that $q = e$. Therefore the kernel is trivial.

For the opposite direction, suppose that the homomorphism we define in the first part of the proof has trivial kernel. As the homomorphism has trivial kernel, then we have that $\varphi \colon G \to S_X$ is injective. Suppose that we have $\varphi (g) = \varphi(h)$ for some $g, h \in G$. Then $\sigma_g = \sigma_h$, for all $x \in X$, $\sigma_g (x) = \sigma_h (x)$ so $g \cdot x = h \cdot x$ and as $\varphi $ is injective, we have $g = h$. Therefore the action is faithfull.
\end{proof}
\begin{theorem}[Orbit-Stabilizer]
	Let $G$ be a group which acts on a finite set $X$ and fix $x \in X$. Then $ |G \cdot x| = [G \colon G_x].$
\end{theorem}

\subsection{Centralizers and Normalizers}
Let $G$ be a group and let $A \subseteq G$.
\begin{defi}
	Define $C_G (A) = \{ g \in G \colon g a g^{-1} = a \; \text{for all $a \in A$} \}$. This subset of $G$ is called the \textbf{centralizer} of $A$ in $G$. Since $g a g^{-1} = a $ if and only if $ga = ag$, $C_G (A)$ is the set of elements of $G$ which commute with every element of $A$. 
\end{defi}
We can clearly see that $C_G (A) \subseteq G$ and $C_G (A) \neq \varnothing $ as $e a e^{-1} = e a = a $ so $e \in C_G (A)$. Now take $x, y \in C_G (A)$. Then $x, y \in G$ and $xax^{-1} = a = yay^{-1}$. Lastly lets consider $xy^{-1}$:$(xy) a (xy)^{-1} = (xy)a(y^{-1} x^{-1}) = x(yay^{-1})x^{-1} = xax^{-1} = a$, and so $xy^{-1} \in C_G (A)$. Therefore the centralizer of $A$ in $G$ is a subgroup of $G$, i.e. $G_G (A) \leq G$.

Now if $G$ is abelian, then we clearly see that $ga = ag$ for all $g \in G$ and $a \in A$, so $C_G (A) = G$. 
\begin{defi}
	Define $Z(G) = \{g \in G \colon gx = xg \; \text{for all}\;  x \in G \}$, the set of elments commuting with all the elements of $G$. This subset of $G$ is called the \textbf{center} of $G$.
\end{defi}
We can make the clear observation that $C_G (G) = Z(G)$ and so we also have that $Z (G) \leq G$. If we do this directly, however, in good spirit, $Z(G)$ is trivially nonempty and obviously a subset of $G$. If we take $x, y \in Z(G)$. Then $yt = ty$ and $xs = sx$ for all $s, t$. Now $et = te$, that is, $y^{-1} y t = y^{-1} t y = t y y^{-1}$, multiplying by $y$ we get $y^{-1} t y^2 = ty$ so $y^{-1} t y = t $ which gives $y^{-1} t = t^{-1}$; thus $y^{-1} \in Z(G)$. Lastly, take some $l \in G$, then $xy^{-1} l = x l y^{-1} = l xy^{-1}$. Therefore $Z(G)$ is a subgroup of $G$.

\begin{defi}
	Define $gAg^{-1} = \{ g a g^{-1} \colon a \in A \}$. Define the \textbf{normalizer} of $A$ in $G$ to be the set \\$N_G (A) = \{ g \in G \colon gAg^{-1} = A \}$. 
\end{defi}
The reason for defining the normalizer in this manner is possibly the main cannon of confusion for this defintion: for the set's needed condition $gAg^{-1} = A$, we're saying that $g a g^{-1} = a^\pp$ for some $a, a^\pp \in A$ and we don't necessarily have that $a = a^\pp$. Note that in the case where they are equivalent however,  then we would have this element $g \in G$ that makes $gag^{-1} = a$ a true statement so $g \in C_G(A)$. Thus we see can predict, with good vision, that $C_G (A) \leq N_G(A)$. Furthermore, we have that $N_G (A) \leq G$. 
\subsection{The Symmetric Group} Let $X$ be any set. Then we define the \textbf{symmetric group} to be the set \\ $\mathfrak{S}_X = \{ \varphi \colon X \to X \colon \varphi \; \text{is a bijection} \}$. You can easily verify that this in fact a group with the group operation being the usual composition of functions. Typically, we deal simply with the set $X = \{ 1, 2, 3, \ldots, n \}$.We assume familiarity with cycle notation.
\begin{prop}
	If the pair of cycles $\alpha= (a_1 \; a_2 \; \cdots \; a_m)$ and $\beta= (b_1 \; b_2 \; \cdots \; b_n)$ have no common entries, i.e. they're disjoint, then $\alpha \beta = \beta \alpha$. 
\end{prop} 
\begin{proof}
	Let $a_i$ be in $\{a_1, a_2, \ldots a_m \}$. Then $\alpha \beta (a_i) = \alpha (\beta (a_i)) = \alpha (a_i) = a_{i+1}$, and $\beta \alpha (a_i) = \beta (\alpha (a_i)) = \beta (a_{i+1}) = a_{i+1}$. Similarily, let $b_i$ be in $\{b_1, b_2, \ldots, b_n \}$. Then $\alpha \beta (b_i) = \alpha (\beta (b_i)) = \alpha (b_{i+1}) = b_{i+1}$, and $\beta (\alpha (b_i)) = \beta (b_i) = b_{i+1}$. 
\end{proof}
\section{Dihedral Group} In general, the \textbf{dihedral group} is the group of symmetries of a regular polygon, where the elements in the groups are represenative of rotations and reflections. The group is defined as $D_{2n} = \langle r, s \colon r^n = s^2 =1, \;  rs=sr^{-1}  \rangle$. We use the convention of denoting the dihedral group as $D_{2n}$ rather than $D_n$ as $| D_{2n} | = 2n$, as opposed to using $D_n$ to denote the $n$-gon (e.g. for the group of symmetries of a triangle, $D_{2(3)} = D_6$ rather than $D_3$). We can note that, taking the geometric point of view, that $s \neq r^i$ for any $i$, as if we have a regular polygon (e.g. take a square) then if we rotate it as many times as we please, then it wouldn't make any sense for it to be equivalent to a roation along any axis (label and work it with a square to help visual aid). Furthermore, we also have that $sr^i \neq sr^j$, for all $0 \leq i, j \leq n- 1$ with $i \neq j$, so a presentation of $D_{2n}$ in terms of looking at its elements, $$D_{2n} = \{ 1, r, r^2, \ldots, r^{n-1}, s, sr, sr^2, \ldots, sr^{n-1} \}.$$
\begin{rem}
	As we've seen in the definition of $D_{2n}$ we have that $rs = sr^{-1}$ for all $s, r$, but we in fact have $r^i s = sr^{-i}$ for all $i \geq 1$. We proceed by induction: the base case is trivial, as we have it by definition. Let $r^k s = sr^{-k}$ for all $k \geq 1$. Then $r (r^k s ) = (r r^k s) = r^{k+1} s = r(sr^{-k}) = (rs) r^{-k} = (sr^{-1})(r^{-k}) = s (r^{-1} r^{-k}) = s^{-1(k+1)}$, and we're done. 

In general, all the elements of $D_{2n}$ have a (unique) representation in the form $s^k r^i$, $k= 0$ or $1$ and $0 \leq i \leq n-1$. If we take $x, y \in D_{2n}$, then $x = s^k r^i$ and $y = s^l r^j$ and for now we make no assumptions about the exponents on $s$. Then $xy = (s^k r^i)(s^l r^j) = s^k (r^i s^l) r^j = s^k (s^l r^{-i}) r^j = s^{k+l} r^{j-i}$. Thus this works, and it is clear that any assumptions upon the exponents $l$ and $k$ would've not changed the outcome. 
\end{rem}

We ca note here that $D_2$ is ismorphic to $Z_2$, which makes $D_2$ abelian, and that $D_4$ is isomorphic to $K_4$, the klein four-group. However, $D_2$ and $D_4$ are the exception as no other dihedral group is abelian beyond these two examples. Moreover, we have that $D_{2n}$ is a subgroup of the symmetric group $\mathfrak{S}_n$ for $n \geq 3$. 
\section{Cosets}
Let $G$ be a group and let $H \leq G$. Then we define a (left) \textit{coset} to be the set $gH = \{ gh \colon h \in H \}$ for each $g \in G$. The element $g$ of the coset $gH$ is said to be a \textit{representative} of the coset. Moreoever, the collection of left cosets is denoted $G/H$. That is to say, $G/H = \{gH \colon g \in G \}$. The common short-hand notation is to denote a represenative in $G/H$ by $\overline{g}$, however, we will omit that largely here as it is perhaps clearer to not use it. To provide some interest, we should also note here that we can say, equivalently, that a left coset is an equivalence class of $G/\sim$ where $g_1 \sim g_2$ if and only if $g_1 = g_2 h$ for some $h \in H$. So, two cosets equivalent if $g_1 H = g_2 H$ as $g_1 h = g_2 l$ for some $h, l \in H$, so $g_1 = g_2 (lh^{-1})= g_2 h_2$ where $h_2 = lh^{-1} \in H$.  
\begin{prop}
	Let $G$ be a group and let $H \leq G$. The left cosets of $H$ in $G$ form a partition of $G$. 
\end{prop}

\begin{proof}
	Let $g \in G$. Since $1 \in H$, it follows that $g \cdot 1 = g \in g H$, so every element of $G$ is in some coset of $H$ and so the union of all cosets is all of $G$. By contrapositive, assume that we have $g_1 H$ and $g_2 H$ and $g_1 H \cap g_2 H \neq \varnothing$. Then we have at least one element in both cosets, so let $x \in g_1 H \cap g_2 H$. Then we may write $x$ as $x = g_1 h_1$ and $x = g_2 h_2$, which means that $g_1 h_1 = g_2 h_2$, and so $g_1 = g_2 h_2 h_1^{-1}$; each element of $g_1H$ has the form $g_1 h$ for some $h \in H$, and $g_1 h = (g_2 h_2 h^{-1}_2) h  = g_2 (h_2 h^{-1}_2) \in g_2 H$. Thus we see that $g_1 H \subseteq g_2 H$, and the other inclusion follows by a similar argument. Therefore $g_1 H = g_2 H$.
\end{proof}
\begin{prop} Let $H \leq G$. Then any two cosets have the same order. 
\end{prop}
\begin{proof}
	Take $g, k \in G$. Then let $\varphi \colon gH \to k H$ by $\varphi \colon gh \mapsto kh$. This map is well defined as if we let $gh = gh^\pp$, then $g^{-1} g h = h^\pp = h$. We can also define the map $\psi \colon kH \to gH$ by $\psi \colon kh \mapsto gh$, and this map is indeed well defined. Now take $gh \in gH$, then $(\psi \varphi)(gh) = \psi (\varphi (gh)) = \psi (kh) = (gh) = \id(gh)$, and similarily, if we take $kh \in kH$ then $\varphi (\psi(kh)) = \varphi (gh) = kh = \id(kh)$. Therefore we have a bijection between the two cosets and thus their cardinalities are the same.
\end{proof}
\begin{coro} Any left coset has the same order as $H$. 
\end{coro}

\begin{defi}
If $H \leq G$ then the \textbf{index} of $H$ in $G$, written $[G \colon H$, is the number of left (or right) cosets of $H$ in $G$. That is, $[G \colon H] = | \{gH \colon g \in G \} |$	
\end{defi}
\begin{rem}
If $G$ is finite, then $[G \colon H ] = |G| / |H|$.	Moreover, if $K \leq H \leq G$ then $[G \colon K ] = [G \colon H ] [H \colon K]$ as we have $$[G \colon H ] [H \colon K] = \frac{|G|}{|H|} \cdot \frac{|H|}{|K|} = \frac{|G|}{|K|} = [G \colon K.$$ We will note here, without proof, that this is also true when considering infinite groups. 
\end{rem}




\subsection{Normal Subgroups}
\begin{defi}
If $G$ is a group and $H \leq G$, and for all $g \in G$ we have $gH = Hg$, then we say that $H$ is a \textbf{normal subgroup} of $G$, and write $H \lhd G$. 	
\end{defi}
\begin{lemma}
	\begin{itemize}
		\item[(i)] Every subgroup of an abelian group $G$ is normal.
		\item[(ii)] A subgroup $H$ is normal in $G$ if and only if for all $g \in G$ and $h \in H$, then $ghg^{-1} \in H$
	\end{itemize}
\end{lemma}
\begin{proof}
	Suppose that $G$ is an abelian group and $H \leq G$. Then if we take $g \in G$ and $gh \in gH$, then $gh = hg \in Hg$. So the equality of sets is obvious. Now for (ii): If we have a normal subgroup $H$ of $G$, then $gH = Hg$, so if we take some $gh \in gH$, then $gh = h^\pp g \in Hg$, so $ghg^{-1} = h^\pp \in H$. For the opposite direction, take $g \in G$ and $h \in H$ and suppose $ghg^{-1} = h^\pp \in H$. Then if $gh = h^\pp g$, so $gh \in Hg$ and $h^\pp g \in gH$, so we can see a containment of sets, i.e. $gH = Hg$ as we chose everything arbitrarily. 
	
	\begin{prop}
		Let $\varphi \colon G \to H$ be a group homomorphism.
		\begin{itemize}
			\item [(i)] $\ker \varphi \lhd G$. 
			\item [(ii)] If $K \lhd H$, then $\varphi^{-1} (K) \lhd G$. 
		\end{itemize}
	\end{prop}
	\begin{proof}
		For (i): $\ker \varphi$ is trivially nonempty, and so take $x, y\in \ker \varphi$. So $\varphi(1) =1 =  \varphi (y y^{-1}) = \varphi (y) \varphi(y^{-1})  = \varphi(y^{-1})$; thus $y^{-1} \in \ker \varphi$. Now $\varphi (xy^{-1}) = \varphi (x) \varphi(y^{-1}) = 1$; thus $xy^{-1} \in \ker \varphi$ and $\ker \varphi \leq G$. Take $g \in G$ and and $h \in \ker \varphi$. Then $\varphi(ghg^{-1} )= \varphi(g)\varphi(h)\varphi(g^{-1}) = \varphi(g)\varphi(g^{-1}) = \varphi(gg^{-1}) = 1$. Therefore $ghg^{-1} \ker \varphi$ and $\ker \varphi \lhd G$. 
		For (ii): Let $K \lhd H$. Then recall that the preimage is defined as $\varphi^{-1}(K) = \{ g \in G \colon \varphi(x) \in K \}$. This set is nonempty as $\varphi(1)  = 1\in K$. Take $x, y \in \varphi^{-1}(K) $. Now $\varphi (1) = 1=  \varphi(y y^{-1}) = \varphi (y)\varphi (y^{-1}) \in K$. Clearly we will have that $xy^{-1} \in K$, so $K \leq G$. Now take $h \in \varphi^{-1} (K)$ and $g \in G$. Then $\varphi (ghg^{-1}) = \varphi(g) \varphi (h) \varphi (g^{-1}) \in \varphi (g) K \varphi(g^{-1}) \subseteq K$, as $K$ is normal. Thus $ghg^{-1} \in \varphi^{-1} (K)$ and $\varphi^{-1} (K) \lhd G$. 
	\end{proof}
	\begin{prop}
		If $H \leq G$ and $[G \colon H] = p$ for some prime $p$, then $H \lhd G$. 
	\end{prop}
	
	
\section{Conjugacy Classes}
In a group $G$, two elements $h$ and $\ell $ are called \textit{conjugate} when $h = g \ell g^{-1}$ for some $g \in G$. Recall that for $H \leq G$, the \textit{conjugate} subgroup of $H$ by a fixed $g \in G$ is $$gHg^{-1} = \{ ghg^{-1} \colon h \in H \}.$$
If we fix some element $\ell \in G$, we may seek to know which, and how many, elements in $G$ can be written as $g \ell g^{-1}$ for some $g \in G$. A trivial observation is that we may write the  $\ell$ itself as a conjugate of $\ell$, that is, $\ell (\ell) \ell^{-1} = \ell (1) = \ell$. We wish to prescribe a name to all these elements that can be written in terms of conjugacy by a fixed element (in our case, $\ell$): we denote the set of all such elements in $G$ by $\cl_G (\ell)$ and it is called the \textit{conjugacy} class of $\ell$. Formally, for an element $q \in G$, its \textit{conjugacy class} is the set of all elements to it, i.e. $\cl_G(x) = \{ g x g^{-1} \colon g \in G \}$. 
\begin{rem}
In any group, $\cl_G (1) = \{ 1 \}$, as $g 1 g^{-1} = gg^{-1} = 1$ for all $g \in G$. Moreover, if $x \in Z(G)$, then $gxg^{-1} = x$ for all $g \in G$, so $\cl_G (x) = \{ x \}$. The converse is true here as well: if a conjugacy class has size $1$, then it is in the center of the group. Moreover, if $G$ is abelian then every elment is it own conjugacy class. 
\end{rem}
\begin{prop}
Conjugacy is an equivalence relation. 	
\end{prop}
\begin{proof}
	In this instance, two elements $x \sim y$ if and only if there is a $g \in G$ such that $x = gyg^{-1}$. Now $x = 1x1^{-1} = x$, so $x \sim x$. Now let $x \sim y$ such that $x = gyg^{-1}$, so then $xg = gy$ and $g^{-1} x g= y$; thus $y \sim x$. Lastly, let $x \sim y$ and $y \sim z$ with $x = gyg^{-1}$ and $y = hzh^{-1}$. Then  $x = g(hzh^{-1})g^{-1} = (gh)z(gh)^{-1}$, and so $x \sim z$. 
\end{proof}
\begin{rem}
	As conjugacy is an equivalence relation, then it forms a partition of $G$. Thus the size of $G$ is the sum of all conjugacy classes, that is, $|G| = \sum_{\ell \in G} |\cl_G (\ell) |$. 
\end{rem}
\begin{defi}
For a group $G$ and $g \in G$, its \textbf{centralizer} $Z(g)$ is the set of elements of $G$ commuting with $G$; that is, 
$$Z(g) = \{ x \in G \colon xg = gx \}$$	
\end{defi}

A clear observation that follows from this definition is that if we intersect all centralizers of fixed elements in $G$, then we get the whole centralizer of the group, i.e. $\bigcap_{g \in G} Z(g) = Z(G)$. 
\begin{theorem}
	For each $g \in G$, its conjugacy class has the same size as the index of its centralizer:
	$$| \cl_G (g) | = [G \colon Z(g) ].$$
\end{theorem}
\begin{coro}
For any finite group $G$,  $$ |G| = |Z(G)| + \sum |\cl_G(x_i)|,$$ where the sum is taken over distinct conjugacy classes of size greater than $1$.
\end{coro}
\begin{proof}
	As we showed earlier, conjugacy classes partition $G$, so $|G| = \sum_{ \ell \in G} |\cl_G ( \ell)|$, and by Theorem 4.1., we have $|G| = \sum_{ \ell \in G} [G \colon Z(g) ] = \sum_{\l \in G } |G|/ |Z(g)|$. If we take some $w \in G$, and $ | \cl_G(w) | = 1$, then we must have that the element in $\cl_G(w) $ is $w$ itself: as every conjugacy class itself is in it as $w = 1 w 1^{-1} = w$, so this is necessary. Moreover, if $\cl_G( w) = \{ w \}$, then we have that $w = gwg^{-1}$, so $w g = gw$, i.e. $w \in Z(G)$. So we can collect all these conjugacy classes with size one (all those $w_i \in G$ such that $Z(w_i) = G$) into a single term, we get $$|G| = |Z(G)| + \sum |\cl_G(x_i)|$$ where the te sum is carried out only over those those conjugacy classes with more than one element. 
\end{proof}
\begin{theorem}
	If $G$ is a finite group then each conjugacy class in $G$ has size dividing $|G|$.
\end{theorem}
% \begin{rem}[Class Equation] For any finite group $G$, $$|G| = |Z (G) | + \sum |\cl_G (x_i)|,$$	where the sum is taken over distinct conjugacy classes of size greater than $1$> 
% \end{rem}
Note here that this is not an immediate consequence of Langrange, as conjugacy classes are not subgroup. A simple way to see this is that the only conjugacy class containging the identity is the identity itself. As, if you had $1 \in \cl_G(x)$ for some $x \neq 1 \in G$, then $1 = gxg^{-1}$ for some $g \in G$, so $1g = g = xg$, so $x = 1$, which is a contradiction and we must have $1 \notin \cl_G (x)$.  
\section{Semi-Direct Products}

\begin{exercise}
	Classify the structure of a group of order $20$.
\end{exercise}
\begin{proof}
	Let $G$ be a group of order $20$. Then $|G| = 20 = 2^2 (5)$, and so by Sylow's Theorem, we have $n_5 \mid 4$ and $n_5 \equiv 1 \Mod{5}$, which implies that $n_5 = 1$, however, $n_2 = 1$ or $n_2 = 5$. Let $K$ be the unique, normal $5$-Sylow and let $H$ be a $2$-Sylow. Then the orders of $H$ and $K$ are relatively prime as $\gcd (5, 4) = 1$ and so their intersection must be trivial, i.e. $H \cap K = \{1 \}$. Moreover, $HK \leq G$, and so $|HK| = (|H| |K|)/ (H\cap K) = 20/ 1 = 20$, and so by the Recognition Theorem, we have that $G \simeq K \rtimes_\varphi H$, where $\varphi \colon H \to \aut (K)$. As $|K| =5$, we have that $K \simeq Z_5$, and as $|H| = 4$, then $H \simeq Z_2 \times Z_2 $ or $H \simeq Z_4$. 
	
	For the case of $H \simeq Z_4$, we have that $\varphi \colon Z_4 \to \aut (Z_5)$, but $\aut( Z_5) \simeq Z_4$, so $\varphi \colon Z_4 \to Z_4$. Let $x$ denote the generator of $Z_4$ and $y$ denote the generator of the codomain $Z_4$. Now we need for the order of $y$ to satisfy specific properties, in particular, $|y^i | = 4/ (\gcd (4, i))$ and we need for $|y^i|$ to divide $4$. Testing for values of $i$ up to $4$, we have that $i \in [[1, 4 ]]$ are all choices that work. Now we have four mappings:
	\begin{align*}
		& \psi_1 \colon Z_4 \to Z_4, \psi_1 \colon x \mapsto y^4 = 1 \\
		& \psi_2 \colon Z_4 \to Z_4, \psi_2 \colon x \mapsto y \\
		& \psi_3 \colon Z_4 \to Z_4, \psi_3 \colon x \mapsto y^2 \\ 
		& \psi_4 \colon Z_4 \to Z_4, \psi_4 \colon x \mapsto y^3 
	\end{align*} 
	The map $\psi_1$ produces the group $Z_5 \times Z_4$. We need to consider the other maps and whether or not they are isomorphic to one another. We do this by recognizing that $\psi_2 $ and $\psi_4$ both have order $4$, as when we map $\psi_2 (x^4) = y^4 = 1$ and $\psi_ 3 (x^4) = (y^3)^4 = y^{12} = 1$. This implies that $Z_5 \rtimes_{\psi_1} Z_4 \simeq Z_5 \rtimes_{\psi_3} Z_4$. Lastly, we have $Z_5 \rtimes_{\psi_3} Z_4$. 
	
	For the case of $H \simeq Z_2 \times Z_2$, we have $\varphi \colon Z_2 \times Z_2 \to Z_4$. We note here that $Z_2 \times Z_2 = \langle a, b \colon a^2 = b^2 = (ab)^2 = 1 \rangle$. Once again, let $y$ denote the generator of $Z_4$. Now we, similarily, need $\varphi$ to satisfy certain conditions: $|y^j| = 4/(\gcd (4, j)$ and $y^j \mid 2$. Thus the only satisfactory values of $j$ for which this works for are $j = 2$ and $4$. So we will mappings: 
	\begin{align*}
		& \alpha_1 \colon Z_2 \times Z_2 \to Z_4, \alpha_1 \colon a,b  \mapsto y^2, 1 \\
		& \alpha_2 \colon Z_2 \times Z_2 \to Z_4, \alpha_2 \colon a,b  \mapsto 1, y^2 \\
		& \alpha_3 \colon Z_2 \times Z_2 \to Z_4, \alpha_3 \colon a,b  \mapsto y^2, y^2 \\
		& \alpha_4 \colon Z_2 \times Z_2 \to Z_4, \alpha_1 \colon a,b  \mapsto 1, 1 
	\end{align*}
	We can immediately notice that $\alpha_4$ gives rise to $Z_4 \times Z_2 \times Z_2$. Moreover, we can see that we will have isomorphisms between $\alpha_1$ and $\alpha_2$, and so $Z_4 \rtimes_{\alpha_1} Z_2 \times Z_2 \simeq Z_4 \rtimes_{\alpha_2} Z_2 \times Z_2$. The nontrivial case that we actually hence have is $Z_4 \rtimes_{\alpha_3} Z_2 \times Z_2$.
\end{proof}

\begin{ex}
Classify the structure of a group of order $28$.	
\end{ex}

\begin{proof}
	Let $G$ be a group of order $28$. Then $|G| = 28 = 14 \cdot 2  = 7 \cdot 2^2$, and so $n_7 \mid 4$ and $n_7 \equiv 1 \Mod{7}$ which meas that $n_7 = 1$. However, $n_2 = 7$ or $1$. Let $K$ denote the unique, normal $7$-Sylow and let $H$ be a $2$-Sylow. Then $|K| = 7$ so $K \simeq Z_6$ and $|H| = 4$ so either $H \simeq Z_4$ or $H \simeq Z_2 \times Z_2$. Furthermore, by the same reasoning as the last exercise, we have $G \simeq K \rtimes_\varphi H$, where $\varphi \colon H \to \aut (K)$, or, rather, $\varphi \colon H \to Z_6$. 
	
	If $H \simeq Z_4$, then we will consider mappings $\psi \colon Z_4 \to Z_5$. Let $Z_4 = \langle x \rangle $ and $Z_6 = \langle y \rangle$. Then we need that $|y^i| = 6/(\gcd(6, i)$ to divide $4$. So the only choices of $i$ that satisfy this are $i = 3$ and $6$. Thus we have the maps 
	\begin{align*}
		&\psi_1 \colon Z_4 \to Z_6, \psi_1 \colon x \mapsto y^6 = 1 \\
		& \psi_2 \colon Z_4 \to Z_6, \psi_2 \colon x \mapsto y^3
	\end{align*}
	The first mapping gives rise to $Z_6 \rtimes_{\psi_1} Z_4 \simeq Z_6 \times Z_4$, and the second mapping simply produces $Z_6 \rtimes_{\psi_2} Z_4$. 
	
	If $H \simeq Z_2 \times Z_2$, then we consider mappings $\alpha \colon Z_2 \times Z_2 \to Z_6$. Once again, we need to find satisfactory conditions for $j$ respective to $|y^j| = 6 / \gcd(6, j)$ such that this divides $2$. In this instance, the only satisfactory choice are $j = 6$ and $3$. Thus we have mappings 
	\begin{align*}
		& \alpha_1 \colon Z_2 \times 2 \to Z_6, \; \alpha_1 \colon a,b  \mapsto 1, 1 \\ 
		& \alpha_2 \colon Z_2 \times Z_2 \to Z_6, \; \alpha_2 \colon a,b \mapsto y^3, y^3 \\ 
		& \alpha_3 \colon Z_2 \times Z_2 \to Z_6, \; \alpha_3  \colon a,b \mapsto 1, y^3 \\
		& \alpha_4 \colon Z_2 \times Z_2 \to Z_6, \; \alpha_2 \colon a,b \mapsto 1, y^3 
	\end{align*}
	The last two mappings can be seen to induce the same group, i.e. $Z_6 \rtimes_{\alpha_3} Z_2 \times Z_2 \simeq Z_6 \rtimes_{\alpha_4} Z_2 \times Z_2$, and we of course have the trivial mapping given by $\alpha_1$, so we also have $Z_6 \times Z_2 \times Z_2$. The other group we have is $K_6 \rtimes_{\alpha_2} Z_2 \times Z_2$. 
\end{proof}

In general, when approaching these kinds of classification problems there's a receipe we can follow, as seen clearly by the last two problems. If we have a group $G$ 


\subsection{Inner Direct Product} 
Suppose that $G$ is a group and $H \triangleright G$, $K \leq G$, and $H \cap K = \{ 1 \}$. We seen in the past that $HK = \{ hk \colon h \in H, k \in K \}$ does indeed form a subgroup of $G$, i.e. $HK \leq G$. Now, since $H \cap K = \{ 1 \}$, then for every $x \in HK$, we have a unique pair $(h, k)$ such that $x = hk$. For now, this uniqueness is unclear to me, so let the assumption that $H \cap K = \{1 \}$ be a black box. Now, if we take some $g_1, g_2 \in HK$ such that $g_1 = h_1 k_1$ and $g_2 = h_2 k_2$ for some $h_1, h_2 \in H$ and $k_1, k_2 \in K$, then $g_1 g_2 = (h_1 k_1) (h_2 k_2) = h_1 k_1 h_2 k_2 = h_1 k_1 h_2 k_1 ^{-1} k_1 k_2 = h_1 (k_1 h_2 k^{-1}) k_1 k_3 = h_1 \varphi_{k_1} (h_2) k_3$, where $\varphi \colon K \longrightarrow \aut (H)$ by $ \varphi \colon k \longmapsto \varphi_k$ and $\varphi$ takes some element $x \in H$ to $kxk^{-1}$, i.e. $\varphi_k  \colon x \lmt kxk^{-1}$. Therefore we have that the maps coincide. 

We can take note of that the fact that $g_1 g_2$ is completely determined by the group homomorphism: 
\begin{align*} \varphi \colon K &\lra \inn (H) \leq \aut (H) \\  
k &\lmt \varphi_k 
\end{align*}
Moreover, given two groups $H$ and $K$, we want to define possible group structures on $H \times K = \{ (h, k) \colon h \in H, k \in K \}$. 
\begin{theorem} Let $H$ and $K$ be two groups. Let $\psi \colon K \lra \inn (H)$ be a group homomorphism. Equip the set $H \times K$ by the following group operation: for $(h_1, k_1), (h_2, k_2)  \in H \times K$, 
\[ (h_1, k_1) \cdot (h_2, k_2) = (h_1 \psi (k_1) (h_2), k_1 k_2 ) .\]
Then $(H \times K, \cdot)$ is a group, and is denoted by $H \rtimes_\psi K$, called the \textbf{semi-direct product} of $H$ and $K$ with respect to $\psi$. 
\begin{proof} As $H$ and $K$ are both groups and are both nonempty as needed, then $H \times K$ is nonempty. Now, suppose we have $(h_1, k_1), (h_2, k_2) \in H \rtimes_\psi K$. Then $(h_1, k_1) \cdot (h_2, k_2) = (h_1 \psi (k_1) (h_2), k_1 k_2 )$, and since $\psi (k_1) (h_2)$ produces an element in $H$ as $\psi$ is itself an automorphism, we have that $(h_1, k_1) \cdot (h_2, k_2) = (h_2 h, k_1 k_2 ) \in H \times K$, where $h = \psi (k_1) (h_2)$. Thus we have closure. Now, the identity element of $H \rtimes_\psi K$ is simply $(1, 1)$. For inverses, we're going to define that for some $(h, k) \in H \rtimes_\psi K$, its inverse is $(h, k)^{-1} = (\psi(k^{-1} )(h^{-1}) , k^{-1})$; so, $(h, k) \cdot (h, k)^{-1} = (h \psi (k)( \psi (k^{-1} ) (h^{-1} )) , yy^{-1} )$. The left side of this is a cluster fuck but its not so bad to actually compute: firstly $\psi (k^{-1} )(h^{-1}) = k^{-1} h^{-1} k$, and so $\psi (k) ( k^{-1} h^{-1} k) = k (k^{-1} h^{-1} k) k^{-1} =  h^{-1}$; thus, $(h, k) \cdot (h, k)^{-1} = (h(h^{-1}) , yy^{-1}) = (1, 1)$. Lastly we need to check associatvity:
\begin{align*}
 ( (h, k) \cdot (x, y) ) \cdot (l, t) &= (h \psi (k) (x), ky) ) \cdot (l, t)  \\
 &= (h kxk^{-1}, ky) \cdot (l, t) \\
 &= ( h kxk^{-1} \psi (ky) (l), kyt ) \\
 &= ( hkxk^{-1} (ky) l (ky)^{-1}, k(yt)) \\
 &= (hkx yly^{-1} k^{-1}, k(yt))  \\
 &= (h \psi(k) (xyly^{-1}) , k(yt)) \\
 &= (h \psi (k) (x\psi(y)(l)), k(yt) \\
  &= (h, k) \cdot (x \psi(y)(l), yt) \\ 
 &= (h, k) \cdot ( (x, y) \cdot (l, t)) ; 
\end{align*} 
we should note here that we were using the fact that $H$ and $K$ are groups to do a lot of the work in their respective slots. Therefore $H \rtimes_\psi K$ is a group. 
\end{proof}
\begin{prop}
\end{prop} 
\end{theorem}


\section{Rings}
\subsection{Euclidean Domains}


\begin{defi} Any function $N \colon R \to \zz_{\geq 0}$ with $N(0) = 0$ is called a \textbf{norm} on the integral domain $R$. If $N(a) > 0$ for all $a \neq 0$ define $N$ to be a positive norm.
\end{defi}

\begin{defi}
	The integral domain $R$ is said to be a \textbf{Euclidean Domain} if there is a norm $N$ on $R$ such that for any two elements $a, b \in R$ with $b \neq 0 $ there exists elements $q, r \in R$ with $$ a = qb + r \; \; \text{with} \; r =0 \; \text{or} \; N(r) < N(b).$$
	The element $q$ is called the \textbf{quotient} and the element $r$ is the \textbf{remainder} of the division. 
\end{defi}
\begin{rem}
Recall that if $D$ is a square free integer, then we can produce the field $\mathbf{Q} (\sqrt{D} ) = \{a +b \sqrt{D} \colon a, b \in \qq \}$. We may observe that $\qq (\sqrt{D} ) \subseteq \cc$, and so we can show that $\qq (\sqrt{D})$ is subring of $\cc$, however, we will omit the proof here as it is easy. In fact, this shows that $\qq (\sqrt{D})$ is an integral domain as any subring of an field is an integral domain. The reason for picking $D$ to be square free ($D$ has prime factorization with exponents at most one, i.e. $D = p_1 p_2 \cdots p_n$ with all $p_i$ prime and $1 \leq i \leq n$) is so that every element in $\qq (\sqrt{D})$ can be written uniquely in the form $a+b \sqrt{D}$. Moreover, this further gives us that if $a, b \neq 0 \in \qq$, then $a^2 - Db^2 \neq 0$, and furthermore, as $(a+b\sqrt{D})(a-b\sqrt{D}) = a^2 - Db^2$, then we can recognize that $(a-b\sqrt{D})/(a^2 - Db^2)$ is the multplicative inverse of $a+b\sqrt{D}$ in $\qq (\sqrt{D})$ if $a+b\sqrt{D} \neq 0$. Thus we have that $\qq (\sqrt{D})$ is a field (called a \textit{quadratic field}).

Continuing our discussion, we can immediately recognize that $\zz [\sqrt{D}] $ is a subring of $\qq (\sqrt{D})$. Now if $D \equiv 4 \Mod{4}$ then we can product the slightly larger subset $$\zz \left [ \frac{1+\sqrt{D}}{2} \right] =  \left \{ a+b \frac{1+ \sqrt{D}}{2} \colon a, b \in \zz \right \}. $$ This, once again, forms a subring of $\qq \sqrt{D}$. We define $\mathcal{O}_{(\qq \sqrt{D})} = \zz [\omega ] = \{a + b \omega \colon a, b \in \zz \}$, where $\omega = \sqrt{D}$ if $D \equiv 2, 3 \Mod{4}$ or $\omega = (1+\sqrt{D})/ 2$ if $D \equiv 1 \Mod{4}$. We call $\mathcal {O}_{ (\qq \sqrt{D} )}$ the \textbf{ring of integers} in the quadratic field $\qq\sqrt{D}$. 

Define the \textit{field norm} $N \colon \qq (\sqrt{D})  \to \qq$ by $N (a+b \sqrt{D}) = (a+b\sqrt{D})(a-b\sqrt{D}) = a^2 - Db^2 \in \qq$, and as $a, b, D \in \zz$, we also have that $N(a+b (\sqrt{D})) \in \zz$, and so we can begin to talk about the Euclidean aspect to this all. We do this by considering $\mathcal{O}_{(\qq \sqrt{D})}$ as they are integral domains with a norm defined by taking the absolute value of the field norm, i.e. $|N(a+b\sqrt{D})|$, so as to satisfy that $N  \colon R \to \zz_{\geq 0 }$. In general, however, $\mathcal{O}_{(\qq \sqrt{D})}$ isn't Euclidean; in fact, $\mathcal{O}_{(\qq \sqrt{D})}$ is only Euclidean, using the norm we just defined, for finitely many choices of $D$, and in, general, for any possible norm on $\mathcal{O}_{(\qq \sqrt{D})}$, and $D<0$, there are once again finitely many choices of $D$ in which it is Euclidean, and it is an open problem to classify when $\mathcal{O}_{(\qq \sqrt{D})}$ is Euclidean, with whatever norm, for $D>0$. 
\end{rem}
\begin{prop}
$\mathcal{O}_{(\qq \sqrt{-1})} 	 = \zz[\sqrt{-1} ] = \zz[i]$ is a Euclidean domain.
\end{prop}
\begin{proof}
	We define the norm $N \colon \zz[i] \to \zz_{\geq 0}$ to be $N (a+bi)= (a+bi)(a-bi) = a^2 +b^2$. Now take $\alpha = a+bi, \beta = c+di \in \zz[i]$ $\beta \neq 0$. Then in the field $\qq (i)$ we have that $$\frac{\alpha}{\beta} = \frac{a+bi}{c+di} = \frac{(a+bi)(c-di)}{c^2+d^2} = \frac{ac-adi+bci+bd}{c^2+d^2} = \frac{(ac+bd) +(bc-ad)i}{c^2+d^2} = \frac{ac+bd}{c^2+d^2} + \frac{(bc-ad)}{c^2+d^2}i.$$ Now, we define in the clear manner, $\alpha/ \beta = r+si$, and clearly $r, s \in \qq$. Let $p$ be an integer closest to the rational number $r$ and let $q$ be an integer closest to the rational number $s$, so that both $|r-p|$ and $|s-q|$ are at most $1/2$. (It may help to think of the rational numberline here.) The division algorithm follows once we have that $$ \alpha = (p+qi) \beta + \gamma \; \text{for some} \; \gamma \in \zz[i] \; \text{with} \; N(\gamma) \leq \frac{1}{2} N(\beta).$$ Let $\theta = (r-p) + (s-q)i$ and set $\gamma = \beta \theta$. Then $\gamma = \alpha - (p+qi) \beta$, so that $\gamma \in \zz[i]$ is a Gaussian integer and $\alpha = (p+qi) \beta + \gamma$. Since $N( \theta) = (r-p)^2 + (s-q)^2$ is at most $1/4 + 1/4 = 1/2$, and as $N$ is multiplicative, we have $N ( \gamma) = N(\beta) N(\theta)$ so $N(\gamma) \leq \frac{1}{2} N(\beta)$. 
	\end{proof}
\begin{prop}
Every ideal in a Euclidean domain is principal. More precisely, if $I$ is any nonzero ideal in the Euclidean domain $R$ then $I = (d)$, where $d$ is any nonzero element of $I$ of minimum norm. 	
\end{prop}
\begin{proof}
	Let $A$ be a Euclidean domain. Then let $J$ be an ideal of $A$. If $J$ is the zero ideal, then $J = (0)$ and we are done. Let $m$ be the nonzero, minimum element of $J$ with respect to the norm, say, $N$ endowed on $A$. We many note here that the set $\{N (a) \colon a \in J \}$ indeed has a minimum element by the Well Ordering of $\zz$; recall that $N(a) \in \zz_{\geq 0}$ as $N \colon A \to \zz_{\geq 0}$. So as $J$ is an ideal then $mx \in J$ for $x \in A$; thus $(m) \subseteq J$. Let $\ell \in J$, and so as $A$ is Euclidean, we may write $\ell = mq + r$ for some $q, r \in A$ and $N(r) < N(m)$. We may rewrite this as $\ell - mq = r$, and as $J$ is an ideal, then $\ell - mq \in J$, but then $N(r) < N(m)$ is smaller so this contradicts the minimality of $N(m)$. We must have that $\ell = mq + 0 = mq$, and so every element in $J$ can be written as $mt$ for some $t \in A$. Thus $J \subseteq (m)$, which means $J = (m)$, and so $A$ is a PID. 
\end{proof}




\subsection{UFD}


\begin{defi}
Let $R$ be an integeral domain.
\begin{itemize}
	\item Suppose $r \in R$ is nonzero and is not a unit. Then $r$ is called \textbf{irreducible} in $R$ if whenever $r = ab$ with $a, b \in R$, at least one of $a$ or $b$ must be a unit in $R$. Otherwise, $r$ is said to be \textbf{reducible}. 
	\item The nonzero element $p \in R$ is called \textbf{prime} in $R$ if the ideal $(p)$ generated by $p$ is a prime ideal. In other words, a nonzero element $p$ is called prime if it is not a unit and whever $p \mid ab$, then either $p \mid a $ or $p \mid b$.
	\item Two elements $a$ and $b$ of $R$ differing by a unit are said to be \textbf{associate} in $R$ \\ (i.e. $a = ub$ for some unit $u$ in $R$). 
\end{itemize}	
\end{defi}
We can note here that the reason we define these things in this manner is that we take the ring $\zz$ our motiviation. For example, the conidition of \textit{primeness} comes from Euclid's lemma in $\zz$. And the notion of irreduciblity is what is associated to a prime in $\zz$. The prime $p = 11$ can only be written in the form $11 = 11 \cdot 1$, and so we need a unit, i.e. $1 \in R$ in order to write $11$ as a product of more than one integer. 
\begin{prop}
	In an integeral domain a prime element is always irreducible.
\end{prop}
\begin{proof}
	Suppose $R$ is an integral domain and let $p \in R$ be prime. If we let $p$ be reducibe, then we can write $p = ab$ where $a, b \in R - R^\times$. This means that $ab \in (p)$ and so either $a \in (p)$ or $b \in (p)$. If we let $a \in (p)$, then $a = pt $ for some $t \in R$, and so $p = ab = (pt ) b$, which means that $1 = bt$ and this is a contradiction as we assume $b \notin R^\times$. Similarily, if we let $b \in (p)$, then $b = ps$ for some $s \in R$. Thus, $p = ab = a (ps)$ which implies that $a \in R^\times$; a contradiction. Therefore we must have that $p$ is irreducible. 
\end{proof}
\begin{prop}
	In a PID a nonzero element is prime if and only if it is irreducible.
\end{prop}

\begin{proof}
	Let $A$ be a PID. As $A$ is a PID then it is also integral domain, by definition, therefore we're done for the forward direction. Now, let $p$ be irreducibe. If $M$ is an ideal containing $(p)$ then $M = (m)$ is prinicipal, i.e. $(p) \subseteq (m)$. So $p = mx$ for some $x \in A$, but as $p$ is irreducible we must have that either $m$ is a unit or $x$ is a unit. If $m \in A^\times$, then $mv = 1$ for some $v \in R^\times$, but this then implies that $1 \in (m)$, and so $(m) = A$. Moreover, if $x \in R^\times$ then $xl = 1$ for some $l \in R^\times$, so $p = p \cdot 1 = p (xl) $ so $p = (mx) = pl x $, and so $ m = pl  $; thus $m \in (p)$ so $(m) \subseteq (p)$, which means $(m) = (p)$. Thus $(p)$ is maximal, and so as $A$ is a PID, we have that $(p)$ is a prime ideal.
\end{proof}

\begin{defi}
A \textbf{Unique Factorization Domain} (UFD) is an integral domain $R$ in which every nonzero element $r \in R$ which is not a unit has the following properties: 
\begin{itemize}
	\item[(i)] $r$ can be written as a finite product of irreducibles $p_i$ of $R$ (not necessarily distinct): $r = p_1 p_2 \cdots p_n$, and 
	\item[(ii)] the decomposition in (i) is \textit{unique up to associates}: namely, if $r = q_1 q_2 \cdots q_m$ is another factorization of $r$ into irreducibles, then $m = n$ and there is some renumbering of the factors so that $p_i$ is associate to $q_i$, that is, $p_i = q_i \ell$ where $\ell \in R^\times$, for $i = 1,2, \ldots n$. 
\end{itemize}	
\end{defi}

\begin{prop}
	In a UFD, a nonezero element is prime if and only if it is irreducible. 
\end{prop}
\begin{proof}
	By the Proposition 5.3., we have the forward direction already done. Let $A$ be a UFD and let $\ell $ be irreducible in $A$ and assume that $\ell \mid ab$ for some $a, b \in A$. Then we may write $ab = \ell k$ for some $k \in A$. Writing $a$ and $b$ as a product of irreducibles, then we have two characterization of $ab$ and so we must have that the irreducible $\ell$ must be associate to one of the irreducibles in either $a$ or $b$. WLOG, let $\ell$ be associate to one of the irreducibles in the factorization of $a$, and so we may write $a = (u \ell) p_1 p_2 \cdots p_n$ for $u \in A^\times$ and some (possibly empty set of) irreducibles $p_1, \ldots, p_n$. Thus $a = \ell (u p_1 p_2 \cdots p_n)$ and so $\ell \mid a$.  
\end{proof}


\subsection{Polynomial Rings}

Recall that the polynomial ring $R[x]$ in the indeterminate $x$ with coefficients from $R$ is the set of all formal sums $a_n x^n + a_{n-1} x^{n-1} + \cdots + a_1 x + a_0$ with $n \geq 0$ and each $a_i \in R$. Addition of polynomials is done ``componentwise": $$\sum_{i=0}^n a_i x^i +\sum_{i=0}^n b_i x^i = \sum_{i = 0}^n (a_i + b_i)x^i. $$
Multiplication may at first seem odd without actually computing what multiplication might look like for some two polynomials: take $f, g \in R[x]$ with $f = a_n x^n + a_{n-1}x^{n-1} + \cdots a_1 x +a_0$ and $g = b_m x^m + a_{m-1} x^{m-1} + \cdots + b_1 x + b_0$. Then 
\begin{align}
	f g &= (a_0 + a_1 x + \cdots +a_{n-1} x^{n-1} + a_n x^n )  (b_0 + b_1 x +\cdots + b_{m-1} x^{m-1} + b_mx^m) \\
	&= (a_0b_0 + a_0 b_1 x + \cdots a_0 b_{m-1} x^{m-1} + a_0 b_m x^m )+ a_1 b_0 x + a_1 b_1 x^2 + \cdots + a_1 b_{m-1} x^m + a_1 b_m x^{m+1} + \cdots  \\ 
	&= a_0 b_0 + (a_0 b_1 + a_1 b_0 )x + (a_0b_2 + a_2 b_0 + a_1 b_1) x^2 + \cdots
\end{align}
This continues going on, however, this multiplication can be described by 
$$fg = \left (\sum_{i= 0}^n a_i x^i \right) \left ( \sum_{i=0}^m b_i x^i \right) = \sum_{k=0}^{n+m} \left (\sum_{i=0}^k a_i b_{k-i} \right) x^k.$$


\begin{prop}
	Let $R$ be an integral domain. Then 
	\begin{itemize}
		\item[(i)] $\deg (p(x) q(x)) = \deg (p(x)) + \deg (q(x)) $ if $p(x), q(x) \in R[x]$ and both polynomials are nonzero, 
		\item[(ii)] $(R[x])^\times = R^\times$, and 
		\item[(iii)] $R[x]$ is an integral domain.
	\end{itemize}
\end{prop}

If we consider localization, i.e. the quotient field of $R[x]$, then the quotient field, say, $K$ of $R[x]$ consists of all quotients $p(x)/k(x)$ where $k(x)$ is not the zero polynomial. The quotient field $K$ is called the field of rational functions in $x$ with coefficients in $R$. 

\begin{prop}
Let $I$ be an ideal of the ring $R$ and let $(I) = I[x]$ denote the ideal of $R[x]$ generated by $I$ (the set of polynomials with coefficients in $I$). Then $$R[x] / (I) \simeq (R/I) [x].$$
In particular, if $I$ is a prime ideal of $R$ then $(I)$ is a prime ideal of $R[x]$. 
\end{prop}
\begin{proof}
	Define the map $\varphi \colon R[x] \to (R/I) [x]$ by taking a polynomial $ \varphi (f) = \varphi (a_n x^n + a_{n-1}x^{n-1} + \cdots a_1 x +a_0) = (a_n + I)x^n + (a_{n-1} + I)x^{n-1} + \cdots + (a_1 + I)x + (a_0 + I)$. This map can be easily realized to be a homomorphism (think of terms simply as $\overline{a_i x^n}$) and surjective. Now if we consider the kernel of the map, then we have $$ \ker (\varphi) = \{ q \in R[x] \colon \varphi (q) = I \}. $$ A polynomial in $R[x]$ that's in the kernel of $\varphi$ can clearly be seen to need coefficients in $I$, and so $\ker (\varphi) = I[x]$. As the map is surjective an a homomorphism, then $R[x]/I[x] \simeq (R/I)[x]$. 
\end{proof}
\begin{ex}
	If we let $R = \zz$ and take the ideal $n\zz = (n)$ of $\zz$. Then the above proposition gets us that $\zz [x] / n\zz [x]  \simeq (\zz / n \zz) [x]$.  Moreover as $\zz / p \zz \simeq \mathbf{F}_p$ is a field then $\zz/ p \zz [x]$ is an integral domain (as well as a Euclidean domain). The last remark could've been also realized another way: $p \zz = (p)$ is a prime ideal of $\zz$ and so $\zz / p \zz$ is an integral domain and by Proposition 5.6. (iii), we have that $(\zz / p \zz) [x]$ is an integral domain. 
\end{ex}
\subsubsection{Polynomial Rings Over Fields} 

\begin{theorem}
	Let $F$ be a field. The polynomial ring $F[x]$ is a Euclidean domain. Specifically, if $a(x)$ and $b(x)$ are two polynomials in $F[x]$ with $b(x)$ nonzero, then there are \textit{unique} $q(x)$ and $r(x)$ in $F[x]$ such that 
	
$$a(x) = q(x) b(x) + r(x) \; \text{with} \; r(x) = 0 \; \text{or} \; \deg(r(x)) < \deg (b(x))$$
\end{theorem}

\begin{coro}
If $F$ is field, then $F[x]$ is a PID and UFD. 	
\end{coro}
\begin{rem}
Recall that $F$ is a field if and only if $F[x]$ is a PID. Furthermore, if $R$ is any commutative ring such that $R[x]$ is a Euclidean domain, then $R$ is a field. 
\end{rem}

\subsubsection{Polynomial Rings that are UFD} Recall that if we have an integral domain $R$ then we can embed it into its field of fraction, say, $F$, so that $R[x] \subseteq F[x]$ is a subring and $F[x]$ is a Euclidean domain (and hence a PID and UFD). A motivation towards using $F[x]$ rather than $R[x]$ is that, well, we have more ready machinery in $F[x]$, and as $R[x] \subseteq F[x]$ then we can approach some questions about $R[x]$ in $F[x]$. For example, if we have $p(x) \in R[x]$. Then $p(x) \in F[x]$ and as $F[x]$ is a UFD we can factor $p(x)$ uniquely into a product of irreducibles in $F[x]$. In general $R[x]$ is not a UFD, and so if $R$ is an integral domain that isn't a UFD, then $R[x]$ cannot be a UFD as well. 

\begin{prop}[Gauss' Lemma] Let $R$ be a UFD with field of fractions $F$ and let $p(x) \in R[x]$. If $p(x)$ is reducible in $F[x]$ then $p(x)$ is reducible in $R[x]$. More precisely, if $p(x) = A(x)B(x)$ for some nonconstant polynomials $A(x), B(x) \in F[x]$, then there are nonzero elements $r,s \in F$ such that $r A(x) = a(x)$ and $sB(x) = b(x)$ both lie in $R[x]$ and $p(x) = a(x) b(x)$ is a factorization in $R[x]$. 	
\end{prop}

\end{document}